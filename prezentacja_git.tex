\documentclass{beamer}
\usepackage{polski}
\usepackage[polish]{babel}
\usepackage[utf8]{inputenc}
\usepackage{default}

\usepackage[style=authortitle,backend=biber]{biblatex}
\addbibresource{bibliografia.bib}

\title{Git}
\author{Marta Sawko}
\usetheme{Singapore}
\usecolortheme{dove}

\begin{document}

\frame{\titlepage}
\begin{frame}{Plan semianrium}
  \tableofcontents[hidesubsections]
\end{frame}

\begin{frame}
 \frametitle{System kontroli wersji}
 \framesubtitle{\textbf{V}ersion \textbf{C}ontrol \textbf{S}ystem}
 Jest to (co?), pozwalający na szeroko rozumiane nadzorowanie zmian w plikach w czasie, np.:
 \begin{itemize}
  \item badania zmian w obrębie konkretnych fragmentów tekstu, jak linijki
  \item powrotu do wcześniejszych wersji 
  \item w pracy grupowej - znajdowanie winnego powstałych zmian
 \end{itemize} 
 Pierwszy pomysł na rozwiązanie - Local Control Version System.
\end{frame}

\section{Centalizacja a rozproszenie}
\subsection{Centralizacja}
\begin{frame}{\textbf{C}entral \textbf{V}ersion Control \textbf{S}ystems}
  Zlokalizowane w jednym miejscu. (tzn gdzie, serwer?) \\
  Posiadając pewne uprawnienia można wiedzieć dokładnie co kto robi.\\
  obrazek, fig 1.2
  \\ Przykładowe CVS\@: \textit{Subversion}, \textit{Perforce}
\end{frame}

\begin{frame}
  Potencjalne wady CVS:\@
  \begin{itemize}
  \item kiedy pada centralny serwer, zaniechana jest praca nad projektem
  \item kiedy padnie sieć[inaczej] współpracownika zaniechana jest jego praca nad projektem
  \item kiedy pada centralny dysk, na którym wszyscy pracują, to tracimy wszystkie dane 
  \item tylko snapshoty (co to znaczy?)
 \end{itemize}
\end{frame}

\subsection{Dystrybucja}
\begin{frame}
 \frametitle{\textbf{D}istributed \textbf{V}ersion \textbf{C}ontrol \textbf{S}ystems}
  Każdy klon jest pełną kopią danych.
  [fig. 1-3]
  \\Przykłady DVCs: \textit{Git, Mercurial}
\end{frame}

\section{Git}
\subsection{Historia}
\begin{frame}
 \frametitle{Jak zrodził się Git?}
GNU, jądra Linuxa, patches, BitKeeper \\
Linus Torvalds założył, że to nowe narzędzie powinno \\
  \begin{itemize}
  \item być dystrybuowalne
  \item unieść jądro Linuxa
  \item wspierać pracę nieliniową
 \end{itemize}
\end{frame}

\subsection{Istota}
\begin{frame}{Czym jest?}
 Tutaj bardzo nie rozumiem:
 Stream of snaphots - nie przetrzymują kolejnej serii plików, ale "linkują".(Dochodzi to pytanie - co to znaczy, że Git patrzy globalnie, nie plikocentralnie, ale projektocentralnie)
 [książka,str.32]\\
 Przeglądanie historii odbywa się na poziomie lokalnym.\\
 Lokalnie odbywa się również \textit{commit}.\\
\end{frame}

\begin{frame}{Trzy fazy pracy Gita}{Committed, modified, staged}
 Wyróżniamy trzy etapy pracy w ramach lokalnego repozytorium
 fig.1-6
\end{frame}


\subsection{Obsługa}

\subsection{Zalety}
\begin{frame}
\frametitle{Zalety}
\begin{itemize}
 \item 
 \item Commit popełnia się offline, a synchronizacja jest nieinteraktywna
\end{itemize}
 
\end{frame}

\begin{frame}[c]{Tu będzie obrazek i referencja}
  \footfullcite{pro_git}
\end{frame}

\subsection{Rozmówki gitowskie}
\begin{frame}
 \frametitle{Savoir-vivre w Gitcie}
\end{frame}

\end{document}
