\documentclass{beamer}
\usepackage{polski}
\usepackage[polish]{babel}
\usepackage[utf8]{inputenc}
\usepackage{default}

\usepackage[style=authortitle,backend=biber]{biblatex}
\addbibresource{bibliografia.bib}

\title{Git}
\author{Marta Sawko}
\usetheme{Singapore}
\usecolortheme{dove}

\usepackage{xcolor}
\usepackage{listings}
\usepackage{caption}
\DeclareCaptionFont{white}{\color{white}}
\DeclareCaptionFormat{listing}{%
  \parbox{\textwidth}{\colorbox{gray}{\parbox{\textwidth}{#1#2#3}}\vskip-4pt}}
\captionsetup[lstlisting]{format=listing,labelfont=white,textfont=white}
\lstset{frame=lrb,xleftmargin=\fboxsep,xrightmargin=-\fboxsep}

\begin{document}

\frame{\titlepage}
\begin{frame}{Plan seminarium}
  \tableofcontents
\end{frame}

\begin{frame}
 \frametitle{System kontroli wersji}
 \framesubtitle{\textbf{V}ersion \textbf{C}ontrol \textbf{S}ystem}
 Jest to (co?), pozwalający na szeroko rozumiane nadzorowanie zmian w plikach w czasie, np.:
 \begin{itemize}
  \item badania zmian w obrębie konkretnych fragmentów tekstu, jak linijki
  \item powrotu do wcześniejszych wersji 
  \item w pracy grupowej - znajdowanie winnego powstałych zmian
 \end{itemize} 
 Pierwszy pomysł na rozwiązanie - Local Control Version System.
\end{frame}

\section{Centalizacja a rozproszenie}
\subsection{Centralizacja}
\begin{frame}{\textbf{C}entral \textbf{V}ersion Control \textbf{S}ystems}
  Zlokalizowane w jednym miejscu. (tzn gdzie, serwer?) \\
  Posiadając pewne uprawnienia można wiedzieć dokładnie co kto robi.\\
  \begin{center}
   \includegraphics[height=0.4\textwidth]{./obrazki/fig-1_2.png}
   \footfullcite{pro_git}
 \end{center}
  Przykładowe CVS\@: \textit{Subversion}, \textit{Perforce}
\end{frame}

\begin{frame}
  Potencjalne wady CVS:\@
  \begin{itemize}
  \item kiedy pada centralny serwer, zaniechana jest praca nad projektem
  \item kiedy padnie sieć[inaczej] współpracownika zaniechana jest jego praca nad projektem
  \item kiedy pada centralny dysk, na którym wszyscy pracują, to tracimy wszystkie dane 
  \item tylko snapshoty (co to znaczy?)
 \end{itemize}
\end{frame}

\subsection{Dystrybucja}
\begin{frame}
 \frametitle{\textbf{D}istributed \textbf{V}ersion \textbf{C}ontrol \textbf{S}ystems}
  Każdy klon jest pełną kopią danych i stanowi oddzielne kompletne repozytorium.\\
  Czynności takie jak przeglądanie historii, \textit{commit} są szybkie, bo nie wymagają połączenia z serwerem.
\\
 Przykłady DVCs: \textit{Git, Mercurial}
\end{frame}

\begin{frame}
  \begin{center}
   \includegraphics[height=0.7\textwidth]{./obrazki/fig-1_3.png}
   \footfullcite{pro_git}
 \end{center} 
\end{frame}


\section{Git}
\subsection{Historia}
\begin{frame}
 \frametitle{Jak zrodził się Git?}
GNU, jądra Linuxa, patches, BitKeeper \\
Linus Torvalds założył, że to nowe narzędzie powinno \\
  \begin{itemize}
  \item być dystrybuowalne
  \item unieść jądro Linuxa
  \item wspierać pracę nieliniową
 \end{itemize}
\end{frame}

\subsection{Istota}
\begin{frame}{Czym jest?}
 Tutaj bardzo nie rozumiem:
 Stream of snaphots - nie przetrzymują kolejnej serii plików, ale "linkują".(Dochodzi to pytanie - co to znaczy, że Git patrzy globalnie, nie plikocentralnie, ale projektocentralnie)
 [książka,str.32]\\
 Przeglądanie historii odbywa się na poziomie lokalnym.\\
 Lokalnie odbywa się również \textit{commit}.\\
\end{frame}

\begin{frame}{Trzy fazy pracy Gita}{Committed, modified, staged}
 Wyróżniamy trzy etapy pracy w ramach lokalnego repozytorium
  \begin{center}
   \includegraphics[width=0.8\textwidth]{./obrazki/fig-1_6.png}
   \footfullcite{pro_git}
 \end{center}
\end{frame}

\begin{frame}{Różnice w rozumieniu zmian w plikach}{Stream of snapshots vs diff}
  \begin{center}
   \includegraphics[width=0.6\textwidth]{./obrazki/fig-1_4.png}\\
   \includegraphics[width=0.6\textwidth]{./obrazki/fig-1_5.png}
   \footfullcite{pro_git}
 \end{center}

\end{frame}

\begin{frame}{Jak Git dostrzega zmiany?}
 Pliki hashuje się przy użyciu algorytmu \textbf{SHA-1}(Secure Hash Algorithm 1, zapis do 160 bitów, wyświetlanych w postaci 16-stkowej) \\
 
 Filozofia pointerów - wskaźnik Head, pochodne branch, pochodne commits.
\end{frame}

\subsection{Branch}
\begin{frame}{Branch}
   \begin{center}
   \includegraphics[width=0.8\textwidth]{./obrazki/fig-3_8.png}
   \footfullcite{pro_git}
 \end{center}
\end{frame}

\subsection{Merge}
\begin{frame}{Merge}
   \begin{center}
   \includegraphics[width=0.8\textwidth]{./obrazki/fig-3_20.png}
   \footfullcite{pro_git}
 \end{center}
\end{frame}

\section{Obsługa Gita}

\subsection{Inicjalizacja repozytorium}
\begin{frame}{Jak stworzyć repozytorium}
 Dwie strategie
 \begin{itemize}
  \item Import istniejącego projektu \\
  \texttt{git init}
  \item Klon istniejącego już repozytorium gitowskiego \\
  \texttt{git clone https://github.com/libgit2/libgit2}
 \end{itemize}
\end{frame}

\subsection{Stage i commit}
\begin{frame}{Stage i commit}{Wyeksponuj i popełnij}
 
\end{frame}

\subsection{Obsługa branch}
\begin{frame}{Branch}{Zrównoleglij pracę}
 Podstawowe komendy:
 \begin{itemize}
  \item \texttt{git checkout -b branch0} - budowanie nowej gałęzi
  \item \texttt{git checkout  branch1} - przełączenie się na gałąź już istniejącą
  \item \texttt{git checkout -d branch1} - usuwanie gałęzi
  \item \texttt{git branch} - sprawdzanie na jakiej jestem gałęzi i jakie w ogóle istnieją [w moim drzewie?]
 \end{itemize}

\end{frame}

\section{Zalety}

\begin{frame}
\frametitle{Zalety}
\begin{itemize}
 \item Posiadanie pełnej historii we własnym repozytorium
 \item Commit popełnia się offline, a synchronizacja jest nieinteraktywna
 \item Dobrze zorganizowana praca na równoległych gałęziach
 \item Periodic explicit object packing
\end{itemize}
 
\end{frame}

\subsection{Rozmówki gitowskie}
\begin{frame}
 \frametitle{Savoir-vivre w Gitcie}
\end{frame}

\end{document}
